\chapter{Introduction}
\section{Motivation}
The efficient numerical solution of time-dependent partial differential equations (PDE) plays an important role in various fields of application such as fluid dynamics \cite{CM79,W01}, electromagnetics \cite{P98,T05} or heat transfer \cite{J03}. Often, those differential equations consist of nonlinear terms which are challenging to solve numerically. Furthermore, in engineering applications, one is often interested in an optimal solution of the considered PDE with respect to a certain objective function. This leads to the mathematical field of PDE-constrained optimization where a cost function is minimized and the PDE is considered as a constraint. As an example, one might imagine the flow of a fluid in a certain domain which is described by the nonlinear, time-dependent Navier-Stokes equations, cf. \cite{C12}. The objective in that context might be to optimize the shape of the domain such that a desired flow behavior is obtained. This approach is, for example, been used in modern airplane design, cf. \cite{BC03,HM03,MP01} and the introductory examples therein.

Since the numerical treatment of modern engineering problems requires a huge computational effort, one is interested in mathematical methods that reduce the dimension of the underlying dynamical system in order to speedup the optimization problem tremendously and, at the same time, lead to a good approximation to the optimal solution of the original full-size problem. This leads to the field of Model Order Reduction (MOR) which is a relatively new field of mathematics, especially when the considered dynamical systems are nonlinear.
\section{Previous work}
Algorithms for MOR are best known in the context of linear control theory. Prominent textbooks on the field of control theory are for instance \cite{M04,Polderman1998,TSH01}, where so-called \textit{input-state-output systems} of the following form are considered,
\begin{equation}
\begin{split}
\label{ABCD}
\dot{x} &= A x + B u, \\
y &= C x + D u ,
\end{split}
\end{equation}
where $u$ is the input, $x$ is the state of the system, and $y$ is the output of the system. The matrices $A,B,C,D$ are assumed to have the appropriate dimensions. Note that \eqref{ABCD} can be considered as a dynamical system for the special case when $C$ equals the identity matrix and $D = 0$. Then \eqref{ABCD} simplifies to $\dot{y} = A y + B u$, and we consider $u$ as the control and $y(u)$ as the solution of the dynamical system that depends on $u$. Dynamical systems of this form arise for instance after spatial discretization of (linear) partial differential equation with a control.

The aim of model order reduction is to derive a system of the form \eqref{ABCD} with a similar input-output relation but a state variable $x$ of much smaller dimension. In the case of a linear system, there exist a wide range of algorithms that obtain a reduced system and that even guarantee a priori error estimates. The most important methods for linear MOR are:
\begin{itemize}
  \item Balanced Truncation,
  \item Moment Matching,
  \item Hankel-norm approximation,
  \item Krylov methods.
\end{itemize}
All of the methods above are discussed in detail in textbooks on linear MOR (cf. \cite{A05,SVR08}) or in lectures notes of courses on that subject \cite{R12}. A brief presentation can also be found in the corresponding literature study for this thesis \cite{B13}.

Nonlinear dynamical systems with control are of the form:
\begin{align}
\label{nonlinDyn}
\dot{y} = f(y) + Bu,
\end{align}
with $f$ nonlinear. There exist relatively new numerical methods for MOR of \eqref{nonlinDyn} which are of current scientific interest. Among them the method which is widely used in practice is the so-called Proper Orthogonal Decomposition (POD) which is explained in detail in \cite{A00_diss,V11}. A detailed historical review of POD can be found in \cite{KGVB05}. Therein, the authors state the correspondence of POD, which is also known as Karhunen-Lo\`{e}ve decomposition (KLD), and the Principal Component Analysis (PCA) \cite{PCA} which is used in statistics. According to \cite{KGVB05}, the first usage of POD for MOR of dynamical system date back to the 1990s, cf. \cite{PODold}. Note that POD does not lead to a reduction of the involved nonlinearity and therefore, although the dimension of POD-reduced models is lower, the complexity of the nonlinear terms remains the same. The Discrete Empirical Interpolation Method (DEIM) proposed by \cite{Cha11,DEIM} in 2010 is an extension of POD that aims to construct reduced systems that do not depend at all on the dimension of the full-order model. DEIM has already been applied to complex dynamical processes, cf. \cite{CS10}, and has shown to lead to a huge gain in computational complexity. Further methods for MOR of nonlinear system are a generalization of balanced truncation to nonlinear systems \cite{L02} as well as a generalization of moment matching \cite{BS,CW}.

Since we want to apply MOR within the framework of optimal control of (nonlinear) partial differential equations, we also want to refer to standard literature from the field of optimal control and PDE-constrained optimization. A theoretical approach to that field is for instance given by \cite{L12,T10} where questions like existence and uniqueness of an optimal solution are considered. A more practical introduction to PDE-constrained optimization is given in \cite {B75,H10,L71} and the technical report \cite{H08}. An introduction to classical optimization methods is for instance given by \cite{NW06,Rao09}.
\section{Research Goals}
The aim of this Master thesis is the evaluation of MOR techniques when applied in the context of optimal control of nonlinear partial differential equations. In more detail, we are interested in the approximation error of the optimal state and the computational benefit when different optimization algorithms are applied to a reduced model obtained via a POD-DEIM scheme. In \cite{V08}, the author presents a general approach of the application of POD to PDE-constrained optimization. As a standard test case, we consider the optimal control of Burgers' equation as in \cite{H08,RK03}. In \cite{KV99}, the authors applied optimal control algorithms to a POD-reduced model of Burgers' equation with good results regarding accuracy and performance of the reduced model. The main contribution of the present work is, however, to extend the approach in \cite{KV99} by applying DEIM to the POD-reduced Burgers' model. Especially for a small viscosity parameter $\nu$, the full-size Burgers' model requires to be of large dimension in order to guarantee numerical stability. Since the POD-DEIM reduced model is completely independent of the full-model dimension, significant computational speedup can be obtained for any value of the parameter $\nu$. We present a comparison of the optimal control of Burgers' equation when POD and POD-DEIM is applied using three different optimization algorithms. The computational benefit of DEIM within the framework of optimal control has been pointed out for each of the considered optimization methods.
\section{Chapter outline}
The thesis work is structured as follows: In Chapter 2 we present an overview of MOR methods for nonlinear dynamical system. This includes the method of POD and its improvement, the DEIM method. We demonstrate the two methods by deriving a POD-DEIM reduced model for the numerical solution of Burgers' equation in the end of \mbox{Chapter 2}. In Chapter 3, we present some methods for PDE-constrained optimization which can be transformed in implicitly constrained optimization problems. We will focus on two classes of algorithms, one that takes only information of the first derivative of the objective function into account (first-order methods) and another approach that also considers information of the Hessian of the objective function (second-order methods). Also Chapter 3 ends with an application of the presented methods to the (full-order) one-dimensional unsteady Burgers' equation. Chapter 4 deals with a detailed comparison of POD and POD-DEIM when applied to the optimal control of Burgers' equation. We present a complete derivation and implementation of both MOR methods and give an algorithm that solves the optimal control problem on both reduced models. Also a detailed consideration of the approximation error and the computational gain is presented for both, a purely POD-reduced model and the model obtained by POD-DEIM. In \mbox{chapter 5} we give an outlook on future research questions that have not been considered in this thesis. 